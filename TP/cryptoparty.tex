%
% Nahim 'Naam' El Atmani
% ven 17 janvier 2014
%

\documentclass[a4paper]{article}
\usepackage[utf8]{inputenc}
\usepackage[T1]{fontenc}
\usepackage[french]{babel}
\usepackage{fixltx2e}
\usepackage{nth}
\usepackage{listings}
\usepackage{tikz}
\usepackage{fancyhdr}
\usepackage[margin=1.0in]{geometry}
\usepackage{hyperref}

\pagestyle{fancy}
\lhead{{\sc{Chiffrement et anonymat}}\\ {\sc Cryptoparty} - 17 jan. 2014}
\rhead{{\small \sc{GConfs}}\\ {\sc Epita}}
\rfoot{\includegraphics[width=0.15\linewidth]{gconfs.png}}

\begin{document}
\begin{center}
{\Large
    {\bf Cryptoparty / Chiffro-fête, party like it's
            \textsc{1984}\protect\footnote{Questions, remarques, et plaintes à
    \url{naam@riseup.net}, (E-mail, XMPP)}
   }
}
\end{center}

\bigskip

\section*{Introduction\protect\footnote{\url{http://www.cryptoparty.in/\#what\_is\_cryptoparty}}}

Durant cette nuit vous allez apprendre à manipuler de nombreux logiciels
présentés précédemment pendant la conférence. Il est préférable d'avoir son
propre matériel pour suivre l'ensemble des manipulations, cependant votre rack
fera l'affaire du moment que vous ne laissez pas derrière vous des informations
aussi sensibles telle que vos paires de clefs \textsc{GPG}. Vous allez donc créer votre
identité au sein du \textsc{wot\protect\footnote{\url{https://en.wikipedia.org/wiki/Web_of_trust}}}
part la création d'une paire de clef GPG particulière. Vous manipulerez
plusieurs clients reposant sur le protocol XMPP afin de mettre en place des
moyens de conversation sécurisés respectant le principe de conversation
\textsc{otr\protect\footnote{\url{https://otr.cypherpunks.ca/otr-codecon.pdf}}}
(i.e confidentielles,officieuses).

\section{GnuPG alias GPG}

\subsection{Création de clefs}

Dans cette section nous allons générer une paire de clefs afin de pouvoir
chiffer fichiers, mail ou tout autre données pour un ou plusieurs
interlocuteurs. En effet le protocole permet d'envoyer une donnée chiffrée à
plusieurs personnes en même temps en spécifiant ces derniers lors du
chiffrement. Nous verons cela dans une prochaine sous-section.
La principale vulnérabilité du protocole réside dans la faiblesse de la clef
maître, si celle-ci est compromise c'est pratiquement un \emph{game over}.
Pour se prevenir d'une telle situation il convient de faire attention à sa clef
maître en l'isolant quand nous n'en n'avons pas besoin, c'est à dire la pluspart
du temps. Nous allons donc dissocier cette clef maître de ses sous-clefs et
nous les réassocierons uniquement en cas de besoin.

Par defaut \textsc{GPG} créer une sous-clef de signature (signing subkey) et
une sous-clef de chiffrement (encryption subkey). Ce que nous allons faire est
d'ajouter une sous clef de signature au trousseau de clef. Cette nouvelle
sous-clef est directement liée à la première sous clef de signature. Ce triplet
de clef constitut votre trousseau de clefs maître. Vous \textbf{devez} en prendre grand
soin si vous ne voulez pas que les conséquences soient désastreuses plus tard.
\\
Pour créer votre clef sous unix procédez comme suit :
\begin{verbatim}
# gpg --gen-key
gpg (GnuPG) 2.0.22; Copyright (C) 2013 Free Software Foundation, Inc.
This is free software: you are free to change and redistribute it.
There is NO WARRANTY, to the extent permitted by law.

Please select what kind of key you want:
   (1) RSA and RSA (default)
   (2) DSA and Elgamal
   (3) DSA (sign only)
   (4) RSA (sign only)
Your selection? 1
\end{verbatim}

Sélectionnez le choix par défault (1). On vous demande par la suite la taille
de la clef, aujourd'hui on considère que 4096bits suffisent mais ne prenez pas
moins. Sachez aussi que plus la clef est grande, plus les opérations de
chiffrements sont longues et couteuses pour le processeur.
\newpage
Vous devez maintenant choisir la durée de vie de votre paire de clef. Il est
important de ne pas choisir une durée de vie illimité, les raisons en sont
nombreuses. Mais nous pouvons citer le fait qu'une clef périmable peut être
renouvelée si cela est fait avant la date de péremption, donc techniquement elle
a aussi une durée de vie illimité tant que vous le voulez vraiment.
Cela renforce également la sécurité de la clef dans le pire des cas possible,
à savoir que votre clef est compromise et vous l'ignorez.

Pour l'avant dernière étape on vous demande d'entrer l'identité principale liée
à la paire de clef. Tout dépend de l'utilisation que vous voulez en faire, il
vous revient le choix d'un pseudonyme (cf le pseudonymat abordé lors de la
conférence) ou d'une identitée bien définie. Sachez cependant que vous aurez
plus de facilité à développer votre \textsc{WOT} par la suite avec une idendité
vérifiable.

\begin{verbatim}
    Real name: Ted Laparty
    E-mail address: ted@uracuire.org
    Comment:
    You selected this USER-ID:
        "Ted Laparty <ted@uracuire.org>"
\end{verbatim}

La prochaine étape est de loin la plus importante puisqu'elle fait partie
intégrante de la force de votre clef maître. Il s'agit de choisir sa phrase
de pass (ou passphrase). Rappelez vous de la méthode
diceware\protect\footnote{\url{http://world.std.com/~reinhold/diceware.html}} %}\protect\footnote{\link{http://bit.ly/1dOjZW4}}}
pour ce faire!
\\
Vous pouvez ajouter une photographie à la clef si vous le souhaitez afin de
faciliter votre identification lors de l'étape de signature de votre clef par
vos interlocuteurs.

\begin{verbatim}
    # gpg --edit-key ted@uracuire.org
    gpg>  addphoto
    Enter JPEG filename for photo ID: /home/Ted/me.jpg
    gpg> save
\end{verbatim}

Dernière chose, il reste quelques optimisations que nous pouvons faire sur
notre clef avant de la distribuer et de l'utiliser. Il s'agit de choisir ses
algorithmes de hash de notre
clef\protect\footnote{\url{http://www.linux-magazine.com/Online/Features/Protect-your-Documents-with-GPG}}

\begin{verbatim}
    # gpg --edit-key ted@uracuire.org
    gpg> setpref SHA512 SHA384 SHA256 SHA224 AES256 AES192 AES CAST5 ZLIB BZIP2 ZIP Uncompressed
    Set preference list to:
        Cypher: AES256, AES192, AES, CAST5, 3DES
        Digest: SHA512, SHA384, SHA256, SHA224, SHA1
        Compression: ZLIB, BZIP2, ZIP, Uncompressed
        Features: MDC, Keyserver no-modify
    Really update the preferences? (y/N) y
    // Entrer votre phrase de passe.
    gpg> save
\end{verbatim}

Maintenant que l'on a ce trousseau maître nous pouvons signer, chiffrer et
déchiffrer des mails. Cependant nous avons discuté de la dangerosité de porter
sur soit sa clef maître dans la vie de tout les jours, c'est pourquoi nous
allons créer une nouvelle sous clef pour signer et dissocier la clef principale
de ce trousseau de clef. Ceci a pour conséquence de limiter la portée du
problème dans un mauvais scénario. C'est précisément ce nouveau trousseau de
clefs que vous emportez sur votre ordinateur portable tandis que vous laissez
votre clef maître dans un/plusieurs périphériques de stockage mis en lieux sûr.
Vous pouvez aussi appliquer les principes de chiffrement de disque pour la
mettre dans un conteneur chiffré, ce qui aurait pour but de renforcer la
sûreté autour de cette clef.

\begin{verbatim}
    # gpg --edit-key ted@uracuire.org
    gpg> addkey
\end{verbatim}
Choisissez la clef RSA pour signer : \textbf{RSA (sign only)}
\newpage
Il est de bon usage de créer un certificat de révocation juste après avoir
fabriqué sa clef de façon à pouvoir en faire une sauvegarde immédiatement
plutôt que d'oublier de le faire et d'être embêter le jour où par malheur nous
en avions besoin.

\begin{verbatim}
    # gpg --output \<ted@uracuire.org\>.gpg_revoke_cert --gen-revoke ted@uracuire.org
\end{verbatim}
Prenez garde, il est d'usage de ne pas sauvegarder votre clef de révocation à
côté de votre clef maître. Pour la simple et bonne raison que si Mallory met la
main sur ce certificat il peut révoquer votre clef, dans ce cas plus personne
ne pourra vous écrire en utilisant votre clef publique. Ainsi c'est toute votre
identité qui s'écroule.
\\
Nous allons maintenant sauvegarder le tout avant de supprimer la clef maitre
pour fabriquer le trousseau de clef portatif.

\begin{verbatim}
    # gpg --export-secret-keys --armor ted@uracuire.org > \<ted@uracuire.org\>.gpg_private_key
    # gpg --export --armor ted@uracuire.org > \<ted@uracuire.org\>.gpg_public_key
\end{verbatim}

Ce sont ces trois précédents fichiers que vous \textbf{devez} sauvegarder!
Une fois cela fait n'oubliez pas de les supprimer correctement (shred)
\\
Supprimons cette clef maître si dangereuse ! Pour cela on exporte d'abord
l'intégralité de nos sous-clefs (3) puis on supprime la sous clef originale qui
nous servait à signer. Finalement on réimporte nos sous-clefs et on dispose
enfin de ce fameux trousseau de clef portable.

\begin{verbatim}
    # gpg --export-secret-subkeys ted@uracuire.org > tmp_subkeys.asc
    # gpg --delete-secret-key ted@uracuire.org
    # gpg --import tmp_subkeys.asc
    # gpg shred --remove tmp_subkeys.asc
\end{verbatim}
Si shred n'est pas installé sur votre système et que pour une raison obscure
vous ne pouviez l'obtenir utilisez un rm simple, ce qui est \emph{fortement
déconseillé}.

Vous pouvez vérifier que tout s'est passé correctement en demandant à \textsc{GPG}
de vous afficher l'ensemble des clefs dans le trousseau de clef (Keyring) :
\begin{verbatim}
    # gpg -K
      /home/bilbo/.gnupg/secring.gpg
      -----------------------------
      sec# 4096R/XXXXXXXX 2014-01-17
      uid Ted Laparty <ted@uracuire.org>
      ssb 4096R/XXXXXXXX 2014-01-17
      ssb 4096R/XXXXXXXX 2014-01-17
\end{verbatim}
Vous noterez que le \textbf{\#} signifie que gpg ne dispose pas de la clef privée correspondante.
C'est ce que nous voulions. Lorsque vous voulez signer la clef de quelqu'un
vous devez aller chercher votre clef maître jalousement gardée, pour le reste
l'utilisation est similaire à la méthode classique.
\newpage
\subsection{Révoquer sa clef GPG}

Dans le cas où celà vous arriverai d'avoir besoin de le faire, voici les
instructions pour récupérer votre trousseau maître et révoquer vos clefs.
Tout d'abord on reconstruit sont trousseau de clef de la façon suivante :
\begin{verbatim}
    # gpg --import /home/Ted/.../\<ted@uracuire.org\>.gpg_private_key
    # gpg --edit-key ted@uracuire.org
    gpg> key 1
    gpg> key 2
    gpg> revkey

    Do you really want to revoke the selected subkeys? (y/N) y
    Please select the reason for the revocation:
      0 = No reason specified
      1 = Key has been compromised
      2 = Key is superseded
      3 = Key is no longer used
      Q = Cancel
    Your decision? 1
    Enter an optional description; end it with an empty line:
    >
    Reason for revocation: Key has been compromised
    (No description given)
    Is this okay? (y/N) y

    gpg>save
\end{verbatim}

Notez que si vous n'êtes pas sur votre ordinateur vous avez aussi besoin d'importer
votre clef publique!

\subsection{Partager de clefs}
Vous avez plusieurs moyens de partager votre paire de clef, il y à bien
entendu la méthode de l'échange en face à face qui reste la meileure et l'upload
sur votre page web ou sur des keyserver. Ces derniers sont des serveurs qui
se rétropropagent les informations à propos des clefs que vous leur fournissez.
Pour être plus clair, si vous envoyer une clef sur le keyserver suivant
\url{gpg.mit.eu} il sera presque instantanément disponible sur de nombreux autres
serveur. Cependant pour débuter ce n'est pas la méthode de choix, car vous
\textbf{ne pouvez pas supprimer} votre clef du serveur, ni même avoir aucun droit de
modification directe dessus. Aussi, si vous avez construit une paire de clef
de durée illimité,
que vous l'avez ensuite uploadé sur un keyserver puis que vous avez finalement
perdu votre certificat de révocation alors celle-ci demeurrera pour toujours
sur les serveur. Ce qui augmente les chances que vos correspondants se trompent
en vous écrivant en l'envoyant à l'adresse pour laquelle vous n'avez plus aucun
contrôle.
Aussi nous ne n'irons pas plus loin dans l'enregistrement des clefs sur les
keyserver pour éviter les erreurs que beaucoup font y compris moi\footnote{\url{http://pgp.mit.edu/pks/lookup?search=naam92160\%40gmail.com\&op=index}}
même. Si vous regardez la note vous pouvez observer que j'ai créer trois clefs
dont deux que j'avais fabriqué il y à un moment pour tester et pour lesquelles
j'ai eu le malheure de ne pas avoir créer de certificat de révocation
immédiatement après leur création.
\\
En revanche pour importer la clef de quelqu'un la démarche est plus facile,
vous devez procéder exactement comme lorsque vous importer votre clef privée
lorsque vous voulez révoquer une sous clef. Depuis les keyserver (cette fois
sans risques) nous devons chercher le destinataire puis l'importer. Pour cela
on procédera comme suit si nous cherchons Ted et que nous connaissons son email :
\begin{verbatim}
    # gpg --search-keys ted@uracuire.org
\end{verbatim}
Le serveur vous retourne les possibles clefs suivant votre recherche et vous
devez lui indiquer le numéro de la clef que vous voulez importer. C'est tout!

\subsection{Chiffrer}
Venons en à ce pourquoi nous avons fait tout ceci. Pour chiffrer un fichier et
ceci peut importe le type (document, image, texte brut...) vous devez
rensseigner un ou plusieurs destinataires se trouvant dans votre trousseau de
clefs.
\end{document}
