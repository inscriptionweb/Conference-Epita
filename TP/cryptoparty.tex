%
% Nahim 'Naam' El Atmani
% ven 17 janvier 2014
%

\documentclass[a4paper]{article}
\usepackage[utf8]{inputenc}
\usepackage[T1]{fontenc}
\usepackage[french]{babel}
\usepackage{fixltx2e}
\usepackage{nth}
\usepackage{listings}
\usepackage{tikz}
\usepackage{fancyhdr}
\usepackage[margin=1.0in]{geometry}
\usepackage{hyperref}

\pagestyle{fancy}
\lhead{{\sc{Chiffrement et anonymat}}\\ {\sc Cryptoparty} - 17 jan. 2014}
\rhead{{\small \sc{GConfs}}\\ {\sc Epita}}
\rfoot{\includegraphics[width=0.15\linewidth]{gconfs.png}}

\begin{document}
\begin{center}
{\Large
    {\bf Cryptoparty / Chiffro-fête, party like it's
            \textsc{1984}\protect\footnote{Questions, remarques, et plaintes à
    \url{naam@riseup.net}, (E-mail, XMPP)}
   }
}
\end{center}

\bigskip

\section*{Introduction\protect\footnote{\url{http://www.cryptoparty.in/\#what\_is\_cryptoparty}}}

Durant cette nuit vous allez apprendre à manipuler de nombreux logiciels
présentés précédemment pendant la conférence. Il est préférable d'avoir son
propre matériel pour suivre l'ensemble des manipulations, cependant votre rack
fera l'affaire du moment que vous ne laissez pas derrière vous des informations
aussi sensibles telle que vos paires de clefs \textsc{GPG}. Vous allez donc créer votre
identité au sein du \textsc{wot\protect\footnote{\url{https://en.wikipedia.org/wiki/Web_of_trust}}}
part la création d'une paire de clef GPG particulière. Vous manipulerez
plusieurs clients reposant sur le protocol XMPP afin de mettre en place des
moyens de conversation sécurisés respectant le principe de conversation
\textsc{otr\protect\footnote{\url{https://otr.cypherpunks.ca/otr-codecon.pdf}}}
(i.e confidentielles,officieuses).

\section{GnuPG alias GPG}

Les instructions suivantes sont données en ligne de commande car c'est la méthode qui
offre le plus de flexibilité dans nos actions. Vous pouvez les executer sous UNIX/Linux comme sous Windows.
Pour UNIX/Linux, gpg est en général déjà disponible. Pour windows vous installez \textsc{gpg4win}\protect\footnote{\url{www.gpg4win.org/}}
en cochant l'option \emph{GPA} ce qui vous permettra d'avoir une interface graphique si la ligne de commande vous rend
allergique. Cependant sachez que vous ne pourrez pas avoir exactement le même trousseau de clef que celui que nous allons générer.
Au premier lancement de \emph{GPA} une fois celui-ci installé il vous est demandé si vous voulez générer une clef de suite. Indiquez que
oui et suivez les instructions. Quand on vous demande si vous souhaitez sauvegarder la clef indiquez oui et faites en une copie. Référez
vous aux instructions en ligne de commande pour plus de détails.

\subsection{Création de clefs}

Dans cette section nous allons générer une paire de clefs afin de pouvoir
chiffer fichiers, mail ou tout autre données pour un ou plusieurs
interlocuteurs. En effet le protocole permet d'envoyer une donnée chiffrée à
plusieurs personnes en même temps en spécifiant ces derniers lors du
chiffrement. Nous verons cela dans une prochaine sous-section.
La principale vulnérabilité du protocole réside dans la faiblesse de la clef
maître, si celle-ci est compromise c'est pratiquement un \emph{game over}.
Pour se prevenir d'une telle situation il convient de faire attention à sa clef
maître en l'isolant quand nous n'en n'avons pas besoin, c'est à dire la pluspart
du temps. Nous allons donc dissocier cette clef maître de ses sous-clefs et
nous les réassocierons uniquement en cas de besoin.

Par defaut \textsc{GPG} créer une sous-clef de signature (signing subkey) et
une sous-clef de chiffrement (encryption subkey). Ce que nous allons faire est
d'ajouter une sous clef de signature au trousseau de clef. Cette nouvelle
sous-clef est directement liée à la première sous clef de signature. Ce triplet
de clef constitut votre trousseau de clefs maître. Vous \textbf{devez} en prendre grand
soin si vous ne voulez pas que les conséquences soient désastreuses plus tard.
\\
Pour créer votre clef sous procédez comme suit :
\begin{verbatim}
# gpg --gen-key
gpg (GnuPG) 2.0.22; Copyright (C) 2013 Free Software Foundation, Inc.
This is free software: you are free to change and redistribute it.
There is NO WARRANTY, to the extent permitted by law.

Please select what kind of key you want:
   (1) RSA and RSA (default)
   (2) DSA and Elgamal
   (3) DSA (sign only)
   (4) RSA (sign only)
Your selection? 1
\end{verbatim}

Sélectionnez le choix par défault (1). On vous demande par la suite la taille
de la clef, aujourd'hui on considère que 4096bits suffisent mais ne prenez pas
moins. Sachez aussi que plus la clef est grande, plus les opérations de
chiffrements sont longues et couteuses pour le processeur.
Vous devez maintenant choisir la durée de vie de votre paire de clef. Il est
important de ne pas choisir une durée de vie illimité, les raisons en sont
nombreuses. Mais nous pouvons citer le fait qu'une clef périmable peut être
renouvelée si cela est fait avant la date de péremption, donc techniquement elle
a aussi une durée de vie illimité tant que vous le voulez vraiment.
Cela renforce également la sécurité de la clef dans le pire des cas possible,
à savoir que votre clef est compromise et vous l'ignorez.

Pour l'avant dernière étape on vous demande d'entrer l'identité principale liée
à la paire de clef. Tout dépend de l'utilisation que vous voulez en faire, il
vous revient le choix d'un pseudonyme (cf le pseudonymat abordé lors de la
conférence) ou d'une identitée bien définie. Sachez cependant que vous aurez
plus de facilité à développer votre \textsc{WOT} par la suite avec une idendité
vérifiable.

\begin{verbatim}
    Real name: Ted Laparty
    E-mail address: ted@uracuire.org
    Comment:
    You selected this USER-ID:
        "Ted Laparty <ted@uracuire.org>"
\end{verbatim}

La prochaine étape est de loin la plus importante puisqu'elle fait partie
intégrante de la force de votre clef maître. Il s'agit de choisir sa phrase
de pass (ou passphrase). Rappelez vous de la méthode
diceware\protect\footnote{\url{http://world.std.com/~reinhold/diceware.html}} %}\protect\footnote{\link{http://bit.ly/1dOjZW4}}}
pour ce faire!
\\
Vous pouvez ajouter une photographie à la clef si vous le souhaitez afin de
faciliter votre identification lors de l'étape de signature de votre clef par
vos interlocuteurs.

\begin{verbatim}
    # gpg --edit-key ted@uracuire.org
    gpg>  addphoto
    Enter JPEG filename for photo ID: /home/Ted/me.jpg
    gpg> save
\end{verbatim}

Dernière chose, il reste quelques optimisations que nous pouvons faire sur
notre clef avant de la distribuer et de l'utiliser. Il s'agit de choisir ses
algorithmes de hash de notre
clef\protect\footnote{\url{http://www.linux-magazine.com/Online/Features/Protect-your-Documents-with-GPG}}

\begin{verbatim}
    # gpg --edit-key ted@uracuire.org
    gpg> setpref SHA512 SHA384 SHA256 SHA224 AES256 AES192 AES CAST5 ZLIB BZIP2 ZIP Uncompressed
    Set preference list to:
        Cypher: AES256, AES192, AES, CAST5, 3DES
        Digest: SHA512, SHA384, SHA256, SHA224, SHA1
        Compression: ZLIB, BZIP2, ZIP, Uncompressed
        Features: MDC, Keyserver no-modify
    Really update the preferences? (y/N) y
    // Entrer votre phrase de passe.
    gpg> save
\end{verbatim}

Maintenant que l'on a ce trousseau maître nous pouvons signer, chiffrer et
déchiffrer des mails. Cependant nous avons discuté de la dangerosité de porter
sur soit sa clef maître dans la vie de tout les jours, c'est pourquoi nous
allons créer une nouvelle sous clef pour signer et dissocier la clef principale
de ce trousseau de clef. Ceci a pour conséquence de limiter la portée du
problème dans un mauvais scénario. C'est précisément ce nouveau trousseau de
clefs que vous emportez sur votre ordinateur portable tandis que vous laissez
votre clef maître dans un/plusieurs périphériques de stockage mis en lieux sûr.
Vous pouvez aussi appliquer les principes de chiffrement de disque pour la
mettre dans un conteneur chiffré, ce qui aurait pour but de renforcer la
sûreté autour de cette clef.

\begin{verbatim}
    # gpg --edit-key ted@uracuire.org
    gpg> addkey
\end{verbatim}
Choisissez la clef RSA pour signer : \textbf{RSA (sign only)}
\newpage
Il est de bon usage de créer un certificat de révocation juste après avoir
fabriqué sa clef de façon à pouvoir en faire une sauvegarde immédiatement
plutôt que d'oublier de le faire et d'être embêter le jour où par malheur nous
en avions besoin.

\begin{verbatim}
    # gpg --output \<ted@uracuire.org\>.gpg_revoke_cert --gen-revoke ted@uracuire.org
\end{verbatim}
Prenez garde, il est d'usage de ne pas sauvegarder votre clef de révocation à
côté de votre clef maître. Pour la simple et bonne raison que si Mallory met la
main sur ce certificat il peut révoquer votre clef, dans ce cas plus personne
ne pourra vous écrire en utilisant votre clef publique. Ainsi c'est toute votre
identité qui s'écroule.
\\
Nous allons maintenant sauvegarder le tout avant de supprimer la clef maitre
pour fabriquer le trousseau de clef portatif.

\begin{verbatim}
    # gpg --export-secret-keys --armor ted@uracuire.org > \<ted@uracuire.org\>.gpg_private_key
    # gpg --export --armor ted@uracuire.org > \<ted@uracuire.org\>.gpg_public_key
\end{verbatim}

Ce sont ces trois précédents fichiers que vous \textbf{devez} sauvegarder!
Une fois cela fait n'oubliez pas de les supprimer correctement (shred)
\\
Supprimons cette clef maître si dangereuse ! Pour cela on exporte d'abord
l'intégralité de nos sous-clefs (3) puis on supprime la sous clef originale qui
nous servait à signer. Finalement on réimporte nos sous-clefs et on dispose
enfin de ce fameux trousseau de clef portable.

\begin{verbatim}
    # gpg --export-secret-subkeys ted@uracuire.org > tmp_subkeys.asc
    # gpg --delete-secret-key ted@uracuire.org
    # gpg --import tmp_subkeys.asc
    # gpg shred --remove tmp_subkeys.asc
\end{verbatim}
Si shred n'est pas installé sur votre système et que pour une raison obscure
vous ne pouviez l'obtenir utilisez un rm simple, ce qui est \emph{fortement
déconseillé}.

Vous pouvez vérifier que tout s'est passé correctement en demandant à \textsc{GPG}
de vous afficher l'ensemble des clefs dans le trousseau de clef (Keyring) :
\begin{verbatim}
    # gpg -K
      /home/bilbo/.gnupg/secring.gpg
      -----------------------------
      sec# 4096R/XXXXXXXX 2014-01-17
      uid Ted Laparty <ted@uracuire.org>
      ssb 4096R/XXXXXXXX 2014-01-17
      ssb 4096R/XXXXXXXX 2014-01-17
\end{verbatim}
Vous noterez que le \textbf{\#} signifie que gpg ne dispose pas de la clef privée correspondante.
C'est ce que nous voulions. Lorsque vous voulez signer la clef de quelqu'un
vous devez aller chercher votre clef maître jalousement gardée, pour le reste
l'utilisation est similaire à la méthode classique.
\newpage
\subsection{Révoquer sa clef GPG}

Dans le cas où celà vous arriverai d'avoir besoin de le faire, voici les
instructions pour récupérer votre trousseau maître et révoquer vos clefs.
Tout d'abord on reconstruit sont trousseau de clef de la façon suivante :
\begin{verbatim}
    # gpg --import /home/Ted/.../\<ted@uracuire.org\>.gpg_private_key
    # gpg --edit-key ted@uracuire.org
    gpg> key 1
    gpg> key 2
    gpg> revkey

    Do you really want to revoke the selected subkeys? (y/N) y
    Please select the reason for the revocation:
      0 = No reason specified
      1 = Key has been compromised
      2 = Key is superseded
      3 = Key is no longer used
      Q = Cancel
    Your decision? 1
    Enter an optional description; end it with an empty line:
    >
    Reason for revocation: Key has been compromised
    (No description given)
    Is this okay? (y/N) y

    gpg>save
\end{verbatim}

Notez que si vous n'êtes pas sur votre ordinateur vous avez aussi besoin d'importer
votre clef publique!

\subsection{Partager de clefs}
Vous avez plusieurs moyens de partager votre paire de clef, il y à bien
entendu la méthode de l'échange en face à face qui reste la meileure et l'upload
sur votre page web ou sur des keyserver. Ces derniers sont des serveurs qui
se rétropropagent les informations à propos des clefs que vous leur fournissez.
Pour être plus clair, si vous envoyer une clef sur le keyserver suivant
\url{gpg.mit.eu} il sera presque instantanément disponible sur de nombreux autres
serveur. Cependant pour débuter ce n'est pas la méthode de choix, car vous
\textbf{ne pouvez pas supprimer} votre clef du serveur, ni même avoir aucun droit de
modification directe dessus. Aussi, si vous avez construit une paire de clef
de durée illimité,
que vous l'avez ensuite uploadé sur un keyserver puis que vous avez finalement
perdu votre certificat de révocation alors celle-ci demeurrera pour toujours
sur les serveur. Ce qui augmente les chances que vos correspondants se trompent
en vous écrivant en l'envoyant à l'adresse pour laquelle vous n'avez plus aucun
contrôle.
Aussi nous ne n'irons pas plus loin dans l'enregistrement des clefs sur les
keyserver pour éviter les erreurs que beaucoup font y compris moi\footnote{\url{http://pgp.mit.edu/pks/lookup?search=naam92160\%40gmail.com\&op=index}}
même. Si vous regardez la note vous pouvez observer que j'ai créer trois clefs
dont deux que j'avais fabriqué il y à un moment pour tester et pour lesquelles
j'ai eu le malheure de ne pas avoir créer de certificat de révocation
immédiatement après leur création.
\\
En revanche pour importer la clef de quelqu'un la démarche est plus facile,
vous devez procéder exactement comme lorsque vous importer votre clef privée
lorsque vous voulez révoquer une sous clef. Depuis les keyserver (cette fois
sans risques) nous devons chercher le destinataire puis l'importer. Pour cela
on procédera comme suit si nous cherchons Ted et que nous connaissons son email :
\begin{verbatim}
    # gpg --search-keys ted@uracuire.org
\end{verbatim}
Le serveur vous retourne les possibles clefs suivant votre recherche (nom,
email, \textsc{KEYID}). Pour l'importer vous devez devez lui indiquer le numéro
de la clef que vous avez choisis parmis les résultats. C'est tout!

Si on vous donne la clef en main propre vous pouvez procéder comme suit:
\begin{verbatim}
    # gpgp --import ted.gpg_public_key
\end{verbatim}
En toute logique n'acceptez pas de clefs envoyées par mail! Du moins pas tant
que ceux cis ne sont pas eux-même chiffrés \textbf{et signés}.
\newpage
\subsection{Chiffrer}
Venons en à ce pourquoi nous avons fait tout ceci. Pour chiffrer un fichier et
ceci peut importe le type (document, image, texte brut...) vous devez
rensseigner un ou plusieurs destinataires se trouvant dans votre trousseau de
clefs. Si vous ne spécifiez pas de fichier en entrée \textsc{GPG} va lire
le texte à chiffrer directement depuis la console. Voici comment chiffrer un
message pour Ted :
\begin{verbatim}
    # gpg --encrypt mon_mail.txt --recipient ted@uracuire.org
\end{verbatim}
Encore une fois l'email peut être remplacé par tout identificateur de clef,
c'est à dire un nom ou un ID. Cependant le nom n'est pas la meilleure solution
bien que la plus simple au premier abord car il n'identifie pas de manière
unique une clef et est plus prompt à faire une erreur sur le destinataire du
mail par la suite.

\subsection{Déchiffrer}
Vous avez reçus un message ou un fichier \textsc{GPG} identifiable par son entête
si celui-ci à été chiffré avec l'option \emph{armor}\protect\footnote{\url{http://gnupg.org/gph/en/manual/r1290.html}}
ou par son extension (asc, gpg, pgp...).

\textsc{GPG} est suffisemment intelligent pour détecter si vous disposez de la
bonne clef dans votre trousseau pour la déchiffrer, inutile donc de chercher
compliqué :

\begin{verbatim}
    # gpg --decrypt file.asc --output file.clear
\end{verbatim}

\subsection{Signer}
La dernière chose que vous devez savoir faire avant d'être opérationel c'est
de signer un message ou une clef. Signer une clef garanti que l'interlocuteur
qui possède la clef que vous voulez signer se trouvant dans votre trousseau est
bien cette personne. C'est à dire que vous faite confiance en cette personne
ainsi qu'en ses capaciter à signer d'autres clefs. Nous nous contenterons de
comment \textsc{GPG} gère de lui même les chaînes de confiance au début. Toutefois
sachez qu'il existe plusieurs niveau de confiances que vous pouvez accorder à une
clef, ce qui permet de faire des réglages plus fin. Par exemple vous avez
totalement confiance dans la clef d'un membre de votre famille puisque vous
l'avez générée avec lui, cependant vous doutez de ses capacités à signer
correctement des clefs. Dans ce cas il existe un niveau de confiance bien adapté
à votre situation.
\\
Pour signer la clef de Ted nous procéderons comme suit :

\begin{verbatim}
    # gpg --edit-key ted@uracuire.org
    gpg> fpr
    gpg> sign
    gpg> save
\end{verbatim}

Ici \emph{fpr} est le diminutif de footprint, il vous permet de vérifier que la
clef que vous avez récupéré a bien la même empreinte que celle que votre
interlocuteur vous à donné à l'oral ou par écrit précédemment.
Ne signez pas si les empreintes ne correspondent pas!
\\
Signer des clefs permet d'étendre sont réseau de confiance, le WOT. Mais comment
faire pour que notre correspondant ai la certitude que la donnée chiffrée que
nous transmettons est bien issue de notre démarche ? Nous allons signer les
documents, comme vous signez un chèque pour attester que vous êtes le possesseur
du chéquier. Pour cela on peut attacher la signature à la donnée ou la séparer
de façon à avoir un fichier de signature et un fichier contenant la donnée.

\begin{verbatim}
    # gpg --output mail.sig --sign mail
\end{verbatim}

Pour attacher la signature au message dans le cas d'un mail par exemple vous
procéderez de la manière suivante :
\begin{verbatim}
    # gpg --clear-sign mail
\end{verbatim}
Notez que le contenu du message n'est pas chiffré !
\newpage
Ce qui aura pour effet de créer un texte de la sorte :
\begin{verbatim}
-----BEGIN PGP SIGNED MESSAGE-----
Hash: SHA1

this is a test
-----BEGIN PGP SIGNATURE-----
Version: GnuPG v1.4.14 (GNU/Linux)

iQIcBAEBAgAGBQJS1u6HAAoJECP40hfpyw3Foo4P/AqnsKC2vdZsgnqqlp6MRIrI
HpDF0zo0cBgZIWu/suh22VCEPfvyND4u2dKz/asD/uu/uNfMW6NbH3D0fVtzy6Sx
lJPXCdkeMApQMCq9kVNDPp377sS+5/xdm70SAPnph0TQS6vTayYTJ/xCmSk7U22b
/k/dzbciu+VqrnKdWJZhKTkRf5ObnirM3uW9RbUp8khPCm5dm89fVqGZtUPDA+IQ
dyTTakOdFA2nziNDrUKByC1DZr+1uQVJJnzYapIi/pp/E3XQAj36mJ7GGz3OmqZC
lALoIliPsMZy9Ax4DG1y+641LKTQLp70Lpxf1+RwoRd3J7DLTu9Zax8MMv7ubbRQ
pdlT8Q9AbtNDmpRKSV2SueP2s+fgzDE+LDQ0I7A0B6owV1KYj/6MGxdYpESKbYUF
2Z1G1o372wI9xFyhkKBNdASFK+NqtXsKNDynNdQk0mohWKpJo9uxWxxiICuyyccY
PTyV2XwPLqiKEljDeS0cFfpC9o8iJSHW1XsioEb7iqYmbvTie+A/otDX7kC9LYXw
Is+NqEfWeT4jhw5vfpO04DOfzBZajs3EScNhra1yWbkKxDmsMA2ZOWyTEkiVahQS
p8hhlOl+lUr7PF+wlr4jRYZ2hNn/wsf+WqNy53nm9V7o6aTBd6/ErvhXCI4LvyNf
hW5YfukiaaIKROvwfzoR
=hNlH
-----END PGP SIGNATURE-----
\end{verbatim}

Pratiquez, envoyez vous des mails, signez vos clefs!

\section{TOR The onion router}
\subsection{Installation}
Pour utiliser tor préférez installer le \textsc{TBB}\protect\footnote{\url{https://www.torproject.org/projects/torbrowser.html.en}},
De cette manière vous ne risquez pas de laisser fuiter votre identité par un plug-in un peu trop bavard.
Si, en revanche vous voulez utiliser tor pour un autre besoin que visiter des services web vous pouvez installer 
Tor seul (Standalone) pour pour \textsc{UNIX/Linux}\protect\footnote{\url{https://www.torproject.org/download/download-unix.html.en}}
ou pour Windows\protect\footnote{\url{http://www.01net.com/telecharger/windows/Securite/anonymat-vie-privee/fiches/35156.html}}.

L'utilisation de TOR sans le bundle se fait comme l'accès à un service au travers d'un proxy. Vous devez configurer vos applications
pour passer par le proxy en locale que Vidalia administre pour vous sous windows. Sous \textsc{unix/linus} il existe \emph{torsocks}\protect\footnote{\url{http://code.google.com/p/torsocks/downloads/list}}
dont l'utilisation est élémentaire :

\begin{verbatim}
    # usewithtor <command> [arguments]
\end{verbatim}
Notez que selon votre installation, l'executable ci-dessus peut porter d'autres noms comme torify.

Pour tester que votre installation fonctionne vous pouvez aller sur des sites comme\\
\url{http://network-tools.com/} et vérifier que votre IP est bien différente de l'habituelle.

\subsection{Les services cachés du réseau onion}
Quand votre connexion passe par les noeuds \textsc{TOR} vous avez accès à ce qu'on
appel les services cachés de \textsc{TOR}. Cela peut-être un accès SSH comme un
serveur Web. Pour vérifier cela diriger vous directement vers le
\emph{hiddenWiki}\footnote{\url{http://kpvz7ki2v5agwt35.onion/wiki/index.php/Main_Page}}
ou la bibliothèque TOR (TOR Library\footnote{\url{http://am4wuhz3zifexz5u.onion/}}).
Cependant la première chose que vous \textbf{devez absolument éviter} sont les sites relatant de
près ou de loin à des affaires illégales ou à du CP\protect\footnote{\url{https://en.wikipedia.org/wiki/Child_pornography}}
car en plus d'être odieusement malsain vous vous exposez sans le savoir à des possible poursuites judiciaire.
\\

Happy browsing !

\section{MAT\protect\footnote{\url{https://mat.boum.org/}}}

\subsection{GUI}

L'interface est ultra-simpliste. Ajoutez simplement des fichiers à nettoyer
avec le boutton \textbf{+}. Pour avoir la liste des fichiers supporté rendez vous sur le
site officiel (\emph{cf la note de section}) où sur la slide correspondante.
Si vous avez la curiosité de regarder le contenu des métas-données du document
double-cliquez sur la ligne correspondante au document. Et n'oubliez pas de
sélectionner l'ensemble des fichiers puis de lancer le nettoyage des métas-données
avec le bouton \emph{clean} correspondant.

\subsection{Ligne de commande}

Pour afficher les métas-données contenues dans un fichier supporté par \textsc{MAT}
procédez de la sorte :
\begin{verbatim}
    # mat -d cvta.docx
\end{verbatim}

Afin de ne pas avoir à vérifier avec l'option \textbf{-c} chaque document dans un répertoir
avant de les nettoyer (\textbf{-d}) un par un vous pouvez simplement procéder
de la façon suivante :

\begin{verbatim}
    # mat -f /path/to/folder/*
\end{verbatim}

Par defaut l'action de \textsc{MAT} sera de supprimer les méta-données d'un fichier
donné en paramètre.

\subsection{Exemple}
Ted Laparty vient de recevoir un e-mail avec une photo de sa copine très compromettante,
il décide de vérifier qui est à l'origine et l'intégrité de cette image gràce à
\textsc{MAT}. Voici ce qu'il obtient en consultant les métas-données de la photographie :\\
\begin{center}
    \includegraphics[width=.7\textwidth]{./materials/dirty}
        \end{center}
Le voilà rassuré, il s'agit d'un montage fait par un de ses collègue de chez darkharbor.
\newpage

\section{Pidgin\protect\footnote{\url{https://pidgin.im/}} + OTR plugin}
Pidgin est un client multi-protocolaire, il peut tout aussi bien gérer vos connections
IRC que GTalk, Jabber (XMPP, facebook,...).
Nous allons l'utiliser ici pour parler avec nos contacts facebook depuis Pidgin
sans avoir lancer un navigateur. Sachez que les serveur de facebook proposant ce
protocol sont identiques en tout points à ceux que vous trouverez ailleurs à la seule
différence que vous êtes explicitement prévenu que l'historique de vos conversations
est sauvegardée par soucis pratique (pour vous).
Si vous êtes intéressé par la création d'un compte sur un serveur XMPP autres que
ceux de facebook vous pouvez aller voir du côté de
DuckDuckgo\protect\footnote{\url{https://duck.co/forum/thread/3703/howto-utiliser-le-service-de-messagerie-publique}}.

\subsection{Le plugin OTR\protect\footnote{\url{https://otr.cypherpunks.ca/}}}
Ce plugin permet d'ajouter à Pidgin les propriétés dont nous avons discuté pendant
la conférence comme la \emph{perfect forward secrecy}.
Il existe un binaire pour l'installer sous windows, rien de plus simple donc.
Pour \textsc{UNIX/Linux} la page de cypherpunks\protect\footnote{\url{https://otr.cypherpunks.ca/index.php\#docs}}
comporte des tutoriels suivant votre système d'exploitation.
\\
Une fois installé il reste une partie de configuration primordiale. Il faut tout
d'abord activer le plugin, pour cela \emph{Tools->Plugins->Cochez le bon}
\\
Cliquez ensuite sur \emph{Config plugin} et cochez les options suivantes :
\begin{itemize}
        \item Enable private messaging
        \item Automatically initiate private messaging
        \item Don't Log \textsc{otr} conversations
        \item show \textsc{otr} button in toolbar
\end{itemize}
Cliquez ensuite sur le bouton \emph{Generate} et validez.
\\
Une fois dans la fenêtre de chat vous pouvez démarrer la session privée en cliquant
sur le bouton \emph{Not private} puis en sélectionnant \emph{Start private conversation}.
Si votre correspondant dispose d'un client prenant en charge ce type de communication il
va répondre automatiquement oui à votre demande et la communication est désormais chiffrée
avec un status de type \textbf{unverified}. Ce qui signifie que vous n'avez pas encore
vérifié l'identité de votre destinataire. Un peu comme si vous communiquiez par email
avec \textsc{GPG} sans avoir signé la clef de votre correspondant. Autrement dit, c'est
tout à fait possible mais non conseillé. Cliquez donc sur le bouton \emph{Unverified},
anciènement \emph{Not private} puis sélectionnez une des trois possibilités qui s'offrent à vous :
\begin{itemize}
    \item Question / answer: C'est le même principe que pour un mot de passe perdu
        auf que vous posez la question, définissez la réponse correcte mais c'est votre
        destinataire qui va devoir y répondre correctement.
    \item Shared secret : Équivalent d'une passphrase.
    \item Manual : Sans vérification, autentification directe. À utiliser dans le cas
        où vous avez la personne au téléphone par exemple.
\end{itemize}

\subsection{Relier Pidgin à votre compte Facebook}
Pour ce faire créer un nouveau compte sur pidgin en tant que \emph{compte existant}.
Le pseudo (\emph{Username}) correspond à votre pseudo facebook, facilement identifiable quand vous
allez sur votre profil (il figure dans la barre d'adresse).
\emph{Domain} doit correspondre à l'adresse du serveur de chat de facebook qui est
\emph{chat.facebook.com}. Ne cochez pas la case \emph{Create this new account on the server}.
\end{document}
