%----------------------------------------------------------------------------------------
\begin{frame}
\begin{center}
\frametitle{Onion routing principles}
TODO: schema explicatif
\end{center}
\end{frame}


%---------------------------------------------------
\begin{frame}
\frametitle{TOR : The Onion Router}
It's an open-source implementation of the principles we just saw supported by
The Tor Project.
\begin{figure}
\includegraphics[width=0.5\linewidth]{./materials/septical_boy}
\end{figure}
\end{frame}
%---------------------------------------------------
\begin{frame}
\frametitle{TOR : The Onion Router}
\begin{block}{Pros}
\begin{itemize}
\item Hide you identity and location, prevents from eyesdropping.
\item Hide you browsing habbits and act like a debrider on the informations that
you're authorized to see.
\item encrypt your (incom|outgo)ing traffic between nodes.
\end{itemize}
\end{block}
\begin{block}{Cons}
\begin{itemize}
\item Slower connexion, forget about downloading big files, torrents
(deanonymize effect) etc...
\item Still vulnerable to some kind of analysis
\\ (timing deduction or infection between applications).
\item entry/exit nodes are vulnerables, no magic here.
\\ (Partial solution if you setup an exit enclaving node)
\end{itemize}
\end{block}
TOR is an anonymity tool, not a security one.
\end{frame}

%---------------------------------------------------
\begin{frame}
\frametitle{If you use it, do it smartly}
\begin{columns}[c]
\column{.5\textwidth}
\begin{itemize}
\item don't use standalone TOR or Vidalia bundlle
\item prefer the use of the TBB (Tor Browser Bundle)
\item or even better : tails (live Debian), in hostile environment
(public places etc)
\end{itemize}
\column{.5\textwidth}
\includegraphics[keepaspectratio,width=\textwidth, height=.8\textheight]{./materials/tbb}
\end{columns}
Try Tor browser launcher for your distribution, that keep TBB updated. Grab-it
from here :\\ https://github.com/micahflee/torbrowser-launcher
\end{frame}
%----------------------------------------------------------------------------------------
\begin{frame}
\begin{center}
\Huge{Sur Internet, si c'est gratuit, c'est vous le produit }
\end{center}
\end{frame}


%----------------------------------------------------------------------------------------
\begin{frame}
\frametitle{Qu'est-ce que le pistage ?}


\begin{block}{Le pistage sur Internet}
\begin{itemize}
\justifying{
\item Le pistage est un terme qui comprend des méthodes aussi nombreuses et variées que les sites web, les annonceurs et d'autres utilisent pour connaître vos habitudes de navigation sur le Web. 
\item  Cela comprend des informations sur les sites que vous visitez, les choses que vous aimez, n'aimez pas et achetez. 
\item Ils utilisent souvent ces données pour afficher des pubs, des produits ou services spécialement ciblés pour vous. 
}
\end{itemize}
\end{block}
\end{frame}


%----------------------------------------------------------------------------------------
\begin{frame}
\frametitle{Comment est-on tracké?}

\justifying{
\begin{block}{Toutes les publicités nous espionnent}
\begin{itemize}
\item Le bouton Like de Facebook : il permet à FaceBook de savoir que vous avez visité ce site, même si vous n'avez pas cliqué sur ce bouton.
\item Même si vous vous êtes correctement déconnecté de Facebook.
\item De même pour le bouton le +1 de Google, les scripts de Google Analytics, 
\item Tous les publicité, Amazon...
\end{itemize}
\end{block}
}
\begin{center}
\includegraphics[scale=0.3] {./materials/Facebook_like.png}
\end{center}
\end{frame}


%----------------------------------------------------------------------------------------
\begin{frame}
\frametitle{L'extension Firefox LightBeam (ex Collusion)}
Cette extension permet de voir en temps réel qui nous traque et les interconnexions qu'a le site actuellement visité avec d'autres sites.
\begin{center}
\includegraphics[scale=0.3] {./materials/Collusion.png}
\end{center}
\end{frame}

%----------------------------------------------------------------------------------------
\begin{frame}
\begin{center}
\Huge{Anomymat et extensions pour Firefox}
\end{center}
\end{frame}


%----------------------------------------------------------------------------------------
\begin{frame}
\frametitle{Noscript}

Bloque tous les trackers associés au site.

\begin{center}

\end{center}
\end{frame}


%----------------------------------------------------------------------------------------
\begin{frame}
\frametitle{Self destructing cookie}

Suppression automatisée des cookies

\begin{center}
\includegraphics[scale=0.4] {./materials/selfdestructingcookie.png}
\end{center}
\end{frame}

%----------------------------------------------------------------------------------------
\begin{frame}
\begin{center}
\Huge{Changer de moteur de recherche}
\end{center}
\end{frame}

%----------------------------------------------------------------------------------------
\begin{frame}
\begin{center}
\frametitle{Duckduckgo - Google tracks you. We don't.}

\url{https://duckduckgo.com/}
\\
\includegraphics[scale=0.4] {./materials/DuckDuckGo.jpg}
\end{center}
\end{frame}

%----------------------------------------------------------------------------------------
\begin{frame}
\begin{center}
\Huge{Et pour plus de sécurité?}
\end{center}
\end{frame}

%----------------------------------------------------------------------------------------
\begin{frame}
\frametitle{HTTPSEverywhere}

Force le passage en https quand celui-ci est proposé par le site.

\begin{center}
\includegraphics[scale=0.4] {./materials/https-everywhere.jpg}
\end{center}

\end{frame}

%----------------------------------------------------------------------------------------
\begin{frame}
\frametitle{Certificate Patrol}
Permet de valider les certificats d'un site (lié à https).
\begin{center}
\includegraphics[scale=0.4] {./materials/CertificatePatrol.png}
\end{center}
\end{frame}

